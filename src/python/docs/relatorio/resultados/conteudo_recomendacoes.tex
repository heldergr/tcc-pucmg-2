\subsection{Análise de conteúdo das recomendações}

Uma coisa difícil em aprendizado não supervisionado é a validação da qualidade dos resultados obtidos. Neste trabalho seria avaliar o quanto faz sentido as recomendações 
de páginsa da Wikipedia para cada post do blog. Com base nos resultados anteriores eu resolvi analisar de forma geral e também pontualmente algumas
recomendações. 

A validação que faço aqui é muito subjetiva, com base no meu entendimento se o conteúdo recomendado é relevante ou não e os critérios
podem variar de post para post). Alguns critérios podem ser semelhança geográfica, semalhança de assunto, tipo de localidade, e por aí vai. Ao longo 

Para dar um exemplo, um dos posts analisados abaixo é de uma visita ao Parque Nacional Denali, no Alasca. O que seria conteúdo relevante relacionado?
Páginas falando do próprio Denali teriam relevância altíssima. Outras falando de outros parques nacionais (tipo de localidade) ou de outros lugares
do Alasca (mesm estado) seriam também considerados relevantes por mim.

\subimport{./}{vinte_mais_menos_frequentes.tex}

\subsubsection{Análise de recomendações com foco nos posts de origem}

Na seção de variabilidade de documentos recomendado vimos que o documento com \textit{id} 7923 (\textit{Jalapão - Alimentação}) foi o que teve o menor número de documentos distintos recomendados
para ele em toda minha bateria de execuções, tendo 43 distintas páginas da Wikipedia recomendadas como a mais semelhante pra ele. Vamos então ver,
dentre este conjunto, quais as páginas que tiveram maior número de recomendações.

Este post fala sobre alimentação em uma atração turística do estado do Tocantins, então sendo assim eu acho que das 5 páginas mais recomendadas para 
ele a primeira, quarta e quinta tem conteúdo relevante, enquanto a segunda e terceira nem tanto.

\begin{center}
    \begin{tabular}{|c|c|}
        \hline
        \textbf{Título documento destino} & \textbf{Recomendações} \\
        \hline
        Política Nacional de Resíduos Sólidos & 62 \\
        \hline
        Horseshoe Bend (Arizona) & 32 \\
        \hline
        Portal:Nova Zelândia/Colabore & 32 \\
        \hline
        Monumento Natural das Árvores Fossilizadas do Tocantins & 26 \\
        \hline
        Projeto Cerrado Verde & 26 \\
        \hline
    \end{tabular}
\end{center}

No outro extremo da quantidade de páginas distintas recomendados para um mesmo post temos o com \textit{id} 6798 
(\textit{Revelando a Foto - Detalhes em Caeté}), com 168 recomendações distintas das 336 feitas para este post. Trata-se de um post atípico
em nosso blog, que fala de como foi tirada um foto em uma passeio de um dia para uma cidade próxima a BH. Mesmo entre as 5 páginas mais recomendadas 
não há uma sequer que eu ache que tenha relevância para o assunto do post.

\begin{center}
    \begin{tabular}{|c|c|}
        \hline
        \textbf{Título documento destino} & \textbf{Recomendações} \\
        \hline
        Bible Belt & 8 \\
        \hline
        Praça Nauro Machado & 7 \\
        \hline
        Chonchi & 6 \\
        \hline
        Rakahanga & 5 \\
        \hline
        Reserva Biológica Poço das Antas & 5 \\
        \hline
    \end{tabular}
\end{center}

Nas análises anteriores, ao analisar o comportamento baseado em parâmetros, alguns posts se destacaram por ter menos variação entre os 
recomendados para eles. Um destes foi o post de \textit{id} 5599 (\textit{Alasca - Visitando o Denali National Park and Preserve}), que teve a menor 
variação de documentos recomendados ao analisar junto os parâmetros \textit{num\_topics} e \textit{alpha}.
Resolvi analisar então as páginas de destino que mais foram recomendadas para este post, ordenadas pelo quantidade de recomendações de cada página.

A tabela a seguir mostra que a página da Wikipedia \textit{Parque Nacional e Reserva de Denali}, com \textit{id} 1403174 foi recomendada 119 vezes 
para o post acima, o que foi satisfatório visto que a página descreve exatamente o parque nacional que é foco do post em questão.

A segunda página mais recomendada é sobre o Monte Denali, que é assunto discutido internamente no post, enquanto que a quinta mais recomendada é 
sobre uma estrada no Alasca que fica próxima ao parque e de certa forma é conteúdo semelhante. 

A terceira e quarta paǵinas que mais se destacam são de parques nacionais americanos. Mesmo não sendo próximos ao Denali não deixa de ser conteúdo
relevante.

\begin{center}
    \begin{tabular}{|c|c|}
        \hline
        \textbf{Título documento destino} & \textbf{Recomendações} \\
        \hline
        Parque Nacional e Reserva de Denali & 119 \\
        \hline
        Monte Denali & 46 \\
        \hline
        Parque Nacional de Yosemite & 41 \\
        \hline
        Parque Nacional de Yellowstone & 17 \\
        \hline
        Dalton Highway & 8 \\
        \hline
    \end{tabular}
\end{center}

Na analisar de variação de posts recomendados considerando o parâmetro \textit{eta}, o post que mais se destacou foi o de \textit{3339} 
(\textit{Lençóis Maranhenses - Lagoa Verde}). Resolvi então analisar as 5 páginas da Wikipedia que mais eram recomendadas para ele, também para
verificar a relevância.

A página da Wikipedia \textit{Reserva da Biosfera do Cinturão Verde de São Paulo} se destaca, com 86 recomendações para este post. Curiosamente 
apesar do número alto e da semelhança encontrada no treinamento, eu acho que as páginas mais relevantes para este post seriam e quarta e a quinta, 
que se referem a locais relacionados natureza e do estado do Maranhão. A quinta página, inclusive, se trata do parque nacional onde se encontra 
a Laguna Verde e se mostra a recomendação mais relevante. Quando lidamos com aprendizado de máquina nem sempre o primeiro resultado pode ser 
o mais relevante para o que queremos.

\begin{center}
    \begin{tabular}{|c|c|}
        \hline
        \textbf{Título documento destino} & \textbf{Recomendações} \\
        \hline
        Reserva da Biosfera do Cinturão Verde de São Paulo & 86 \\
        \hline
        Reserva Lagoa São Paulo & 35 \\
        \hline
        Lagoa Dom Helvécio & 20 \\
        \hline
        Área de Proteção Ambiental das Reentrâncias Maranhenses & 13 \\
        \hline
        Parque Nacional dos Lençóis Maranhenses & 9 \\
        \hline
    \end{tabular}
\end{center}

Um outro post que achei interessante analisar foi o de \textit{14565}(\textit{Vídeo - Episódio Salar de Atacama}), que conforme vimos em análise 
anterior teve baixa variação de páginas recomendadas no cenário de testes com menor número de tópicos (22). Ao analisar as páginas que mais lhe foram
recomendadas eu achei o resultado muito interessante pois dos 10 listados abaixo 7 são do Atacama, região do assunto do post. A primeira da lista
não é do Atacama mas também fala de vulcão, assunto tratado no conteúdo de origem. Mesmo os dois últimos, que seriam os menos relevantes, ainda 
são interessantes para o assunto.

\begin{center}
    \begin{tabular}{|c|c|}
        \hline
        \textbf{Título documento destino} & \textbf{Recomendações} \\
        \hline
        Osorno (vulcão) & 54 \\
        \hline
        Salar de Atacama & 51 \\
        \hline
        Gêiseres de Tatio & 37 \\
        \hline
        Vulcão Peña Blanca & 20 \\
        \hline
        San Pedro de Atacama & 20 \\
        \hline
        Puna atacamenha & 19 \\
        \hline
        Lascar (vulcão) & 11 \\
        \hline
        Vale do Salinas & 9 \\
        \hline
        Cerro Catedral & 6 \\
        \hline
        Montanhas Rochosas Canadianas & 6 \\
        \hline
    \end{tabular}
\end{center}

\subimport{./}{conteudo_foco_recomendado.tex}

\subimport{./}{analise_semelhanca.tex}
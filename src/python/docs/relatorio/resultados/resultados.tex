\section{Resultados}

A análise dos resultados da recomendação de páginas da Wikipedia com base em posts do blog Nerds Viajantes foi um dos grandes desafios deste trabalho,
senão o maior. O problema é que não há uma forma de verificar a qualidade das recomendações que não envolva subjetividade. A alternativa 
que tive foi fazer uma validação manual de parte do conteúdo, além de analisar estatística e graficamente parte dos resultados. Mas mesmo esta 
análise numérica não é suficiente para avaliar a eficiência do treinamento e o mecanismo de recomendação.

\subimport{./}{plano_execucao.tex}
\subimport{./}{comportamento_variacao_distancias.tex}
\subimport{./}{variabilidade_documentos_recomendados.tex}
\subimport{./}{topico_dominante.tex}    
\subimport{./}{conteudo_recomendacoes.tex}

\subsection{Datasets de resultados}

Foram disponibilizados no repositório do projeto três \textit{datasets} com resultados das recomendações.

O arquivo \href{https://github.com/heldergr/tcc-pucmg-2/tree/main/src/python/docs/relatorio/resultados/resources/resultados_dataset.csv}{resultados\_dataset.csv}
contém o dataset completo. Tem a página recomendada para cada post na fonte de dados de origem e cada parâmetro utilizado.

Contém os campos: \textbf{id\_cenario, fonte\_origem, id\_documento\_origem, titulo\_documento\_origem, fonte\_destino, id\_documento\_destino, 
titulo\_documento\_destino, distancia\_destino, num\_topics, passes, eta, alpha, cenario\_wp}.

O arquivo \href{https://github.com/heldergr/tcc-pucmg-2/tree/main/src/python/docs/relatorio/resultados/resources/top5_posts_por_recomendado.csv}{top5\_posts\_por\_recomendado.csv}
contém um dataset reduzido contendo os cinco posts da base de origem para os quais uma página da Wikipedia foi recomendada como semelhante com
mais frequência.

Contém os campos: \textbf{titulo\_documento\_origem, titulo\_documento\_destino, Recomendacoes}.

Já o arquivo \href{https://github.com/heldergr/tcc-pucmg-2/tree/main/src/python/docs/relatorio/resultados/resources/top5_recomendacoes_por_post.csv}{top5\_recomendacoes\_por\_post.csv}
contém um dataset reduzido contendo as cinco páginas da Wikipedia que mais foram recomendadas para cada post da base de origem como semelhante.

Contém os campos: \textbf{titulo\_documento\_origem, titulo\_documento\_destino, Recomendacoes}.
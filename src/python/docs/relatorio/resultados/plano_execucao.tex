\subsection{Plano de execução de testes}

Para executar o levantamento de documentos semelhantes além de usar os posts do blog como fonte origem de dados de treinamento e as páginas da 
Wikipedia como fonte de dados de comparação eu resolvi variar os seguintes parâmetros, com a descrição dos valores utilizados.

\begin{itemize}
    \item \textbf{topics}: 57, 45, 72, 67, 39, 22, 101
    \item \textbf{eta}: 0.005, 0.05, 0.5, 1
    \item \textbf{alpha}: 0.01, 0.1, 1
    \item \textbf{passes}: 2, 10
\end{itemize}

Combinando os 378 posts da fonte de origem com estes possíveis valores eu obtive 378 * 7 * 4 * 3 * 2 = \textbf{63504 resultados}. Este número 
foi efetivamente multiplicado por 2 porque fiz os testes uma vez para todas páginas da Wikipedia e outra execução apenas para páginas contendo 
apenas páginas que após a limpeza de dados básica tinha tamanho maior ou igual a 40 tokens. O resultado destas duas execuções foi colocado 
em \textit{collections} separadas no MongoDB e neste relatório são usados os termos como \textbf{full} para referenciar o conjunto completo e 
\textbf{gte40} para o conjunto limitado pelo tamanho.

A execução dos testes foi feita através do módulo \textbf{Executor}, explicado no apêndice (COLOCAR LINK INTERNO?). O código para execução deste 
módulo é o seguinte:

\begin{lstlisting}[language=Python, style=mystyle, frame=lines, caption=Código fonte: Treinamento de modelo usando LDA]]
from main.especificacao_testes import EspecificacaoTestes
from main.executor import ExecutorTreinamento
from repository.resultado import ResultadoRepo

# Cria especificação de testes
filename = 'especificao_completa.yml'
especificacao_testes = EspecificacaoTestes(filename)
cenarios = especificacao_testes.get_testes()

# Executa treinamento
executor = ExecutorTreinamento(cenarios)
resultado_df = executor.executar_treinamento()

# Grava resultados no MongoDB
documentos = resultado_df.to_dict('records')
resultado_repo = ResultadoRepo()
resultado_repo.insert_many(documentos)
\end{lstlisting}

A lista de testes a serem executados foi determinadas em arquivo \textit{yml}, formato escolhido pela facildade de definir variações de valores
de forma declarativa. Abaixo está a listagem do arquivo utilizado (que está no repositório do projeto).

\lstinputlisting[style=mystyle, frame=lines, caption=Especificação de execução de testes]{resultados/especificacao.yml}
\section{Limpeza de texto}

Nem sempre utilizamos o texto inteiro quando queremos extrair conhecimento deste texto, que no nosso caso significa encontrar tópicos com suas palavras e 
encontrar documentos que estes tópicos estão presentes. Muitas palavras não são interessantes, como por exemplo conjuntos de palavras que não possuem 
significado mas influenciam o treinamento de modelos, como as conhecidas stopwords. 

A limpeza de texto consiste então de remover dos nossos documentos 
este conjunto de palavras que não somente não agregam ao treinamento como atrapalham o seu funcionamento adequado. Parte da limpeza é comum a todas as 
fontes de dados e parte é específica por fonte de dados escolhida.

\subsection{Limpeza comum}

\begin{itemize}
    \item Remoção de tags HTML
    \item Remoção de pontuações
    \item Remoção de stopwords: contempla a remoção de stopwords gerais da linguagem dos textos, em nosso caso português
    \item Remoção de stopwords adicionais: é uma tarefa que será executada para todas as fontes de dados mas o conjunto de stopwords adicionais é diferente para cada fonte de dados
    \item Stemming
\end{itemize}

\subsection{Posts Nerds Viajantes}

Obtenção de posts do banco de dados do blog Nerds Viajantes.

\subsection{Wikipedia}

Limpeza de textos obtidos da Wikipedia.

\subsection{Melhores Destinos}

Obtenção de posts com conteúdo de promoções do blog Melhores Destinos.

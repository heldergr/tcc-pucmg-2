\section{Coleta de dados}

Parte dedicada a documentação de coleta de dados para treinamento.

\subsection{Posts Nerds Viajantes}

Obtenção de posts do banco de dados do blog Nerds Viajantes.

\subsection{Wikipedia}

Obtenção de páginas da Wikipedia para treinamento.

\subsection{Verbos em língua portuguesa}

Como parte do trabalho eu precisei da lista de verbos em língua portuguesa para analisar o levantamento de tópicos em documentos após remoção de verbos, 
que no contexto deste trabalho poderiam ser pouco relevantes ou até mesmo atrapalhar na definição dos tópicos.

A lista de verbos foi recuperada fazendo webscraping do site \url{https://www.conjugacao.com.br/verbos-populares}. Foi utilizada o pacote requests do Python 
para fazer a requisição das páginas e o pacote Beautiful Soap para extrair o conteúdo do resultado html retornado.

O total de verbos contidos nesta página é de 5000 e para que fossem retornados todos foram necessárias 50 requisições (são 100 verbos em cada página), 
uma para cada página de conteúdo no site.

Os verbos são gravados localmente em uma collection do MongoDB para que possam ser utilizados mais de uma vez sem necessidade de download a cada execução.
Cada documento gravado no MongoDB tem dois campos, o verbo original e o verbo \textit{stemmed}, conforme exemplo abaixo para o verbo \textit{falar}.

\begin{lstlisting}
    {
        'verbo': 'falar',
        'verbo_stemmed': 'fal'
    }
\end{lstlisting}

\begin{comment}
\subsection{Melhores Destinos}

Obtenção de posts com conteúdo de promoções do blog Melhores Destinos.    
\end{comment}

\subsection{Cálculo de coerência}

Um dos parâmetros mais importantes para o algoritmo LDA é o número de tópicos escondidos que devem ser extraídos do corpus de treinamento.

Sendo o LDA um algoritmo não supervisionado é muito difícil validar se o conjunto de tópicos selecionados para um determinado conjunto de documentos
é útil e faz sentido. Não há uma lista de tópicos corretos previamente selecionados para comparar com os resultados obtidos. Uma forma de medir a 
eficiência do número de tópicos escolhidos é a \textbf{coerência}, que é uma avaliação quantitativa da qualidade dos tópicos aprendidos para um determinado 
conjunto de documentos.

Um conjunto de afirmações ou fatos é considerado coerente, se apoiarem um ao outro. Assim, um conjunto de fatos coerente pode ser interpretado 
em um contexto que cobre todos ou a maioria dos fatos. Um exemplo de conjunto de fatos coerentes é “o jogo é um esporte de equipe”, 
“o jogo é jogado com uma bola”, “o jogo exige grandes esforços físicos”.

A coerência de um tópico mede o grau de semelhança semântica entre as palavras que mais contribuem para a definição daquele tópico.
Esta medida ajuda a distinguir entre os tópicos que são \textbf{semanticamente interpretáveis} e os tópicos que são 
\textbf{artefatos de inferência estatística}.

Há várias medidas de coerência e a que escolhi foi a \textbf{c\_v}. Na lista de referências há artigos que explicam esta e outras 
medidas de coerência. O que importa é que um maior valor indica maior coerência.

\subsubsection{Análise de coerências}

Conforme podemos ver código fonte na seção seguinte, há alguns parâmetros importantes para o cálculo da coerência:

\begin{itemize}
    \item \textbf{documents}: lista de documentos que compôem nosso corpus
    \item \textbf{num\_topics}: número de tópicos para os quais queremos calcular a coerência
\end{itemize}

Apesar de não ser explicitamente utilizados no código, os parâmetros \textbf{alpha} e \textbf{eta} mencionados anteriormente são também importantes mas neste cálculo inicial
optei por mantê-los em seu valor padrão, que é 1/número de tópicos. 

A lista de documentos contempla o conjunto completo de textos de nossa fonte de dados de origem. O número de tópicos eu variei entre 5 e 120, de um 
a um. O resultado gerado foi um arquivo csv contendo os seguintes campos:

\begin{itemize}
    \item num\_topics: número de tópicos no Treinamento
    \item coherence: coerência para aquele número de tópicos
    \item tempo\_gasto: tempo total gasto no cálculo de coerência para o número de tópicos
\end{itemize}

Com base no resultado dos cálculos das coerências para o número de tópicos foi gerado o gráfico a seguir, onde já podemos ver que a coerência para 
de crescer e de certa forma e começa a oscilar para cima e para baixo sem grandes variações a partir de um determinado número de tópicos.

\includegraphics{treinamento/resources/coerencia_vs_topicos.png}

Ordenando os valores das coerências nós obtemos as dez maiores e seus respectivos tópicos, conforme vemos abaixo ordenando começando pela maior.

Esta lista da quantidade de tópicos com maior coerência foi muito importante porque dela eu tirei os números de tópicos que eu escolhi para 
variar o parâmetro \textit{num\_topics} na execução do ajuste do modelo e comparação entre os documentos.

\begin{center}
    \begin{tabular}{ |c|c| }
        \hline
        \textbf{Número de tópicos} & \textbf{Coerência} \\ [0.5ex]
        \hline
        57 & 0.4187279439804192 \\
        \hline
        45 & 0.4120513470103957 \\
        \hline
        72 & 0.4118140382511174 \\
        \hline
        67 & 0.4084148429857527 \\
        \hline
        39 & 0.4083294457749056 \\
        \hline
        22 & 0.4050952035903351 \\
        \hline
        66 & 0.4047401968259075 \\
        \hline
        101 & 0.4036451569534363 \\
        \hline
        69 & 0.4027847252786048 \\
        \hline
        26 & 0.4026587678111619 \\
        \hline
    \end{tabular}
\end{center}

Abaixo segue uma listagem com o código fonte usado para o cálculo das coerências para um conjunto de documentos e intervalo de números de tópicos.

\subsubsection{Código fonte}

\lstinputlisting[language=Python, style=mystyle, frame=lines, caption=Código fonte: Cálculo de coerência de tópicos]{resources/coerencia.py}
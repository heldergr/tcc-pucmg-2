\section{Treinamento de modelo}

Treinamento de modelo para sugestão de posts para determinados conteúdos.

\subsection{LDA}

LDA é um algoritmo de aprendizagem não supervisionado que associa tópicos a documentos. Documentos são como textos e cada um pode conter mais de um tópico.
Tópicos são formados por palavras e uma mesma palavra pode fazer parte de mais de um tópico, ou seja, uma mesma palavra contribui para a formação de vários 
tópicos. Os tópicos são descobertos durante o treinamento do modelo mas a quantidade de tópicos deve ser especificada a priori.

Após o treinamento do modelo um documento tem uma distribuição discrete de tópicos e um tópico tem uma distribuição discreta de palavras. 

Como exemplo temos... COLOCAR AQUI EXEMPLO DE UM DOCUMENTO, SEUS TÓPICOS E AS PALAVRAS QUE COMPOEM O TÓPICO.

\begin{itemize}
    \item DÚVIDA: DEVO EXPLICAR COM MAIS DETALHES A IMPLEMENTAÇÃO DO LDA? ACREDITO QUE NÃO...
\end{itemize}

Há vários tipos de uso para o algoritmo LDA, como entender melhor o tipo de documento um determinado conjunto de palavras (notícias, artigo na wikipedia, 
negócios), quantificar as palavras mais usadas e mais importantes em um texto ou mesmo encontrar semelhanças e recomendações de documentos. 

Neste trabalho eu usei LDA para encontrar semalhança entre textos de acordo com os tópicos do modelo treinado e fazer recomendações com base em textos novos.

LDA não tem boa performance com documentos curtos, como tweets, por exemplo, pois ele infere parâmetros a partir da observação de palavras e se não há palavras
suficientes não há as condições necessárias para um bom aproveitamento.

LDA é um algoritmo que trabalha com modelos do tipo bag of words, ou seja, não há importância na ordem das palavras. Outros algoritmos funcionam bem com sentençãs
estruturadas.

\subsubsection{Hiperparâmetros}

\begin{itemize}
    \item $\alpha$: Um valor baixo indica que documentos contém poucos tópicos contribuindo para os mesmos
    \item $\eta$: Um valor baixo indica que tópicos contém poucas palavras contribuindo para os mesmos, enquanto em um valor alto pode haver maior sobreposição 
    de palavras entre tópicos diferentes
\end{itemize}

\subimport*{./}{analise_coerencia.tex}

\subimport{./}{analise_topicos.tex}
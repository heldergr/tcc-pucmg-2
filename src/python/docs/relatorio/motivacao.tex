\section{Motivação}

Hoje em dia a geração de conteúdo acontece de forma muito rápida. Acompanhar tudo que é gerado de forma manual é impraticável, senão impossível, e é
preciso automatizar determinadas tarefas. Uma das áreas onde a automatização pode ser feita é no processamento de textos e diversas informações podem ser 
extraídos dos textos que são escritos e compartilhados na internet, assim como em mídias privadas também.

Uma área que se destaca neste contexto é o processamento de linguagem natural (PLN), que consiste em um conjunto de técnicas que aplicadas a textos
permitem que diversas tarefas sejam executadas sobre esses textos. Uma destas tarefas é processar este conteúdo e identificar contextos dentro dele, 
permitindo que possamos levantar algumas informações como tópicos que relacionam estes documentos. A identificação deste contexto pode não ser simples
porque a mesma informação pode ser escrita de formas distintas utilizando linguagem natural.

\begin{itemize}
    \item velocidade de geração de conteúdo - Ok
    \item Geração de tráfego
    \item time to market
    \item sugestão de conteúdo para enriquecer textos
    \item sugestão de novas leiturs
    \item encontrar cópia do próprio
\end{itemize}

Natural language is messy, ambiguous and full of subjective interpretation, and sometimes trying to cleanse ambiguity reduces the language to an 
unnatural form.